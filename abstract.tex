%%% -*-LaTeX-*-
%%% This is the abstract for the thesis.
%%% It is included in the top-level LaTeX file with
%%%
%%%    \preface    {abstract} {Abstract}
%%%
%%% The first argument is the basename of this file, and the
%%% second is the title for this page, which is thus not
%%% included here.
%%%
%%% The text of this file should be about 350 words or less.

This dissertation presents an exploration into multi-language program verification using the SMACK software verification toolchain, with a focus on integrating the Rust programming language.
%
Initially, we extend SMACK to support verification of Rust programs by leveraging its use of the low-level virtual-machine intermediate-representation (LLVM IR).
%
We then build upon this foundation to add support for verifying programs written in additional languages targeting LLVM IR.
%
By extending SMACK to support verification of mixed-language programs, we aim to support scenarios where programs in multiple languages (e.g., C and Rust) interact.
%
We apply these extensions to prove correctness properties of programs written in these programming languages, and we prove correctness of programs written in multiple languages.
%
Properties we prove include showing the correctness of the use of libraries between programming languages and equivalence of code written in different programming languages.

More specifically, there are several problems that motivate this work.
%
Rust programs cannot ensure safety of so-called \textit{unsafe code}, which is code that the compiler cannot show to be free of memory errors.
%
SMACK is used to show that an interface using unsafe functionality exposes a safe abstraction.
%
Rust programs can be proven to have correct behavior such as in an information-control-flow implementation.
%
Multi-language extensions to SMACK are used to check the correctness of a library written in Rust against a reference implementation written in C.
%
This exercise found that the Rust library was mostly correct, but we discovered two bugs in its implementation.
%
Finally, SMACK is applied to prove the equivalence of a number of functions in a Rust-language port of a math library originally written in C.
%
In this work, SMACK shows the correct implementation of most ported functions and identifies some bugs.

Together, this work on SMACK transforms it from a software verifier for C-language programs, into a software verifier capable of verifying programs written in multiple programming languages and programs that use code written in different programming languages.
%
These extensions to SMACK enable it to prove correct program behavior of programs that use more than one programming language and to prove equivalence of programs ported between programming languages.