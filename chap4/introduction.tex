\vspace{-2.5em}
\section{Introduction}
Rust is a modern programming language focused on safety and native performance.
%
Due to Rust's novelty, it has been reliant on libraries written in other programming languages, often C, to provide many core functions.
%
However, languages such as C do not have the same safety guarantees provided by Rust and are thus unsafe to use inside Rust.
%
Many libraries provide safe Rust bindings to such C language libraries.
%
But this requires the library developer to completely contain any potential unsafety within their library.
%
Additionally, the foreign code used in this process may not be as optimizable as native Rust code, such as allowing inlining, and may inhibit debugging of such programs from Rust.

An alternative option is to instead port the library code from the source language to Rust.
%
This has the upside of making the library native for Rust programs, and limits or eliminates unsafe code used inside the library.
%
Nonetheless, such ports may not always be correct because of subtleties related to the source programming language.
%
This can be exacerbated by the usage of floating-point arithmetic, which introduces additional subtleties that may not be as well known.
%
Moreover, languages may not have complete support for floating-point arithmetic, which may lead to bugs that are difficult to detect within the language itself.

Formal equivalence checking can provide a solution here because it can show that the original-language program matches the behavior of the program ported to the target programming language.
%
In this work, we apply the SMACK (please refer to \cref{sec:smackground}) software verifier to the Rust-language port of the \texttt{musl} C-language math library.
%
SMACK is a software verifier that can work with both C and Rust language programs at once.
%
We give our experiences using SMACK to show the equivalence of these programs and our extensions to the SMACK verifier to help show equivalence of more programs.